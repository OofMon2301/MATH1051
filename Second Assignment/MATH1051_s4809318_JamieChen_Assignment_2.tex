\documentclass[12pt]{article}
\usepackage[paper=letterpaper,margin=2cm]{geometry}
\usepackage{amsmath}
\usepackage{amssymb}
\usepackage{amsfonts}
\usepackage{newtxtext, newtxmath}
\usepackage{enumitem}
\usepackage{titling}
\usepackage{calculator}
\usepackage{polynom}
\usepackage[colorlinks=true]{hyperref}

\setlength{\droptitle}{-6em}

% Enter the specific assignment number and topic of that assignment below, and replace "Your Name" with your actual name.
\title{Assignment \# 2: MATH1051}
\author{Jamie Chen\\ \text{Student Number:} \texttt{48093189} \\ \text{Semester 2, 2023}}
\date{\today}

\begin{document}
\maketitle
% Question 1
\section*{Question 1}
\begin{enumerate}[leftmargin=\labelsep]
    \item Determine if the series are divergent or are absolutely or conditionally convergent:

    % Question 1a

    \begin{enumerate}
        \item $\displaystyle{\sum_{n=0}^{\infty} \frac{1}{({n^2+1})^{3/2}}}$
        \begin{itemize}[label={}]
            \item 
            \item This series is convergent by the comparison test. We can compare this series to the series $\displaystyle{\sum_{n=0}^{\infty} \frac{1}{n^3}}$ which is a convergent p-series. Since $\displaystyle{\frac{1}{({n^2+1})^{3/2}} \leq \frac{1}{n^3}}$ for all $n \geq 1$, by the comparison test, the series is convergent.
            \item 
        \end{itemize}

    % Question 1b

        \item $\displaystyle{\sum_{n=0}^{\infty} \frac{2^{\sqrt{n}}}{3^n}}$
        \begin{itemize}[label={}]
            \item
            \item This series is absolutely convergent by the ratio test. We can use the ratio test to determine the convergence of this series. 
            \item We can see that $\displaystyle{\lim_{n \to \infty} \left| \frac{2^{\sqrt{n+1}}}{3^{n+1}} \cdot \frac{3^n}{2^{\sqrt{n}}} \right| = \lim_{n \to \infty} \left| \frac{2^{\sqrt{n+1}}}{2^{\sqrt{n}}} \cdot \frac{3^n}{3^{n+1}} \right| = \lim_{n \to \infty} \left| \frac{2^{\sqrt{n+1}-\sqrt{n}}}{3} \right|}$. 
            \item Since $\displaystyle{\lim_{n \to \infty} \left| \frac{2^{\sqrt{n+1}-\sqrt{n}}}{3} \right| = 0}$, by the ratio test, the series is absolutely convergent.
            \item 
        \end{itemize}

    % Question 1c

        \item $\displaystyle{\sum_{n=0}^{\infty} (-1)^n \frac{\ln (n)}{n}}$
        \begin{itemize}[label={}]
            \item 
            \item This series is conditionally convergent by the alternating series test. We can use the alternating series test to determine the convergence of this series. We can see that $\displaystyle{\lim_{n \to \infty} \frac{\ln (n)}{n} = 0}$. Since $\displaystyle{\frac{\ln (n)}{n}}$ is a decreasing function for all $n \geq 1$, by the alternating series test, the series is conditionally convergent.
            \item 
        \end{itemize}

    \end{enumerate}

    \newpage 
    % Question 2
    \section*{Question 2}
    \item Determine the radius of convergence.
    
    % Please add working out with the solution

    \begin{enumerate}
        % Question 2a


        \item $\displaystyle{\sum_{n=0}^{\infty} (2n+1)(2x)^{2n}}$
        \begin{itemize}[label={}]
            \item We can calculate this using the ratio test first. The ratio test is:
            \begin{equation*}
                \displaystyle{\lim_{n \to \infty} \left| \frac{(2(n+1)+1)(2x)^{2(n+1)}}{(2n+1)(2x)^{2n}} \right|}
            \end{equation*}
            \item We can simplify the ratio test.
            \begin{equation*}
                \begin{array}[2ex]{r@{~=~}l}
                    \frac{a_{n+1}}{a_n} & \displaystyle{\frac{(2(n+1)+1)(2x)^{2(n+1)}}{(2n+1)(2x)^{2n}}} \\
                    & \displaystyle{\frac{2(n+1)}{2n+1} \cdot (2x)^2} \\
                \end{array}
            \end{equation*}
            Taking trhe limit of this simplified ratio test, we can see that
            \begin{equation*}
                L = \lim_{n \to \infty} \left| \frac{2(n+1)}{2n+1} \cdot (2x)^2 \right| = 4x^2
            \end{equation*}
            Setting $L<1$ for convergence, we can see that
            \begin{equation*}
                \begin{split}
                    4x^2 &< 1 \\
                    x^2 &< \frac{1}{4} \\
                    x &< \frac{1}{2} \\
                \end{split}
            \end{equation*}
            Therefore, the radius of convergence is $\displaystyle{\frac{1}{2}}$.
        \end{itemize}

        % Question 2b

        \item $\displaystyle{\sum_{n=0}^{\infty} n! x^n}$
        \begin{itemize}[label={}]
                \item For finding the radius of convergence, we use the ratio test. The ratio test is:
                \begin{equation*}
                    \lim_{n \to \infty} \left| \frac{a_{n+1}}{a_n} \right| = \lim_{n \to \infty} \left| \frac{(n+1)! x^{n+1}}{n! x^n} \right|
                \end{equation*}
                \item We can simplify the ratio test by cancelling out the $n!$ terms.
                \begin{equation*}
                    \begin{split}
                        \lim_{n \to \infty} \left| \frac{(n+1)! x^{n+1}}{n! x^n} \right| &= \lim_{n \to \infty} \left| \frac{(n+1) \cdot n! \cdot x^{n+1}}{n! \cdot x^n} \right| \\
                        &= \lim_{n \to \infty} \left| (n+1) x \right| \\
                    \end{split}
                \end{equation*}
                \item Taking the limit of the simplified ratio test, we can see that
                \begin{equation*}
                    \lim_{n \to \infty} \left| (n+1) x \right| = \infty
                \end{equation*}
                \item Since the limit is infinity, it means that the series is divergent for all $x \neq 0$. Therefore, the radius of convergence is $0$.
            \end{itemize}
    \end{enumerate}

    \newpage
    % Question 3

    \section*{Question 3}

    \item Calculate the following indefinite integrals:
    \begin{enumerate}
        % Question 3a
        \item $\displaystyle{\int x \sqrt{1-x^2} \, dx}$
        \begin{itemize}[label={}]
            \item We can start by substituting $u = 1-x^2$.
            \begin{equation*}
                \int xu^2 \, dx
            \end{equation*}
            \item We can rearrange the equation to solve for $dx$.
            \begin{equation*}
                du = -2x \, dx
            \end{equation*}
            \item We can substitute $u$ and $dx$ into the integral.
            \begin{equation*}
                = \int -\frac{1}{2} u^{1/2} \, du
            \end{equation*}
            \item We can then do the integration for the whole equation.
            \begin{equation*}
                = -\frac{1}{2} \int u^{1/2} \, du
            \end{equation*}
            \begin{equation*}
                = -\frac{1}{2} \cdot \frac{2u^{3/2}}{3} + C
            \end{equation*}
            \item We can then substitute $u$ back into the equation.
            \begin{equation*}
                = -\frac{1}{2} \cdot \frac{2(1-x^2)^{3/2}}{3} + C
            \end{equation*}
        \end{itemize}

        \newpage
        % Question 3b
        \item $\displaystyle{\int \sqrt{x} \, \ln(x^2) \, dx}$
        \begin{itemize}[label={}]
            \item We can start by simplifying the integral:
            \begin{equation*}
                = 2 \int x^{1/2} \, \ln(x) \, dx
            \end{equation*}
            \item We can then start by integrating by parts, in which the formula is:
            \begin{equation*}
                \int uv' = uv - \int u'v
            \end{equation*}
            \item We let $u=ln(x)$ and $v=x^{1/2}$.
            \begin{equation*}
                u' = 1/x \quad \text{and} \quad v' = \frac{2}{3} x^{3/2}
            \end{equation*}
            \item We can then substitute $u$, $u'$, $v$, and $v'$ into the formula.
            \begin{equation*}
                \begin{split}
                    \int uv' &= uv - \int u'v \\
                    &= \frac{2x^{3/2}\ln(x)}{3} - \int \frac{2x^{1/2}}{3} \, dx \\
                \end{split}
            \end{equation*}
            \item We can then integrate the second part of the equation.
            \begin{equation*}
                = \frac{2x^{3/2}\ln(x)}{3} - \frac{4x^{3/2}}{9} + C
            \end{equation*}
            \item We times the integral by $2$ to get the final answer.
            \begin{equation*}
                = \frac{2x^{3/2}\ln(x)}{3} - \frac{8x^{3/2}}{9} + C
            \end{equation*}
            \item Therefore, the final answer is $\displaystyle{\frac{2x^{3/2}\ln(x)}{3} - \frac{8x^{3/2}}{9} + C}$.
        \end{itemize}

        \newpage
        % Question 3c
        \item $\displaystyle{\int \frac{x^2}{x^2+6x+8} \, dx}$
        \begin{itemize}[label={}]
            \item We can start by simplifying the integral and bring the integral of 1 to the front.
            \item This can be achieved through polynomial long division.
            \begin{center}
                \polylongdiv{x^2}{x^2+6x+8}
            \end{center}
            \item
            \item We can then split the integral into two parts by using the remainder of the polynomial long division.
            \begin{equation*}
                \begin{split}
                    \int \frac{x^2}{x^2+6x+8} \, dx &= \int \frac{x^2+6x+8-6x-8}{x^2+6x+8} \, dx \\
                    &= \int 1 \, dx - \int \frac{6x+8}{x^2+6x+8} \, dx \\
                    &= \int 1 \, dx - 2 \int \frac{3x+4}{x^2+6x+8} \, dx \\
                \end{split}
            \end{equation*}
            \item We can further simplify the second integral.
            \begin{equation*}
                \begin{split}
                    \int \frac{3x+4}{x^2+6x+8} \, dx &= \int \left( \frac{3(2x+6)}{2(x^2+6x+8)} - \frac{5}{x^2+6x+8} \right) \, dx \\
                    &= 3 \int \frac{x+3}{x^2+6x+8} \, dx - 5 \int \frac{1}{x^2+6x+8} \, dx \\
                \end{split}
            \end{equation*}
            \item We can then use the substitution method to solve the first integral by letting $u=x^2+6x+8$.
            \begin{equation*}
                du = (2x+6) dx
            \end{equation*}
            \item We can then substitute $u$ and $du$ into the first integral.
            \begin{equation*}
                \frac{1}{2} \int \frac{1}{u} \, du
            \end{equation*}
            \item We can then integrate the first integral and simplify.
            \begin{equation*}
                \begin{aligned}
                    \frac{1}{2} \int \frac{1}{u} \, du &= \frac{1}{2} \ln(u) + C \\
                    &= \frac{1}{2} \ln(x^2+6x+8) + C \quad \text{since } u=x^2+6x+8 \\
                \end{aligned}
            \end{equation*}
            \item We now solve the second integral by partial fractions.
            \begin{equation*}
                = \int \left( \frac{1}{2(x+2)} - \frac{1}{2(x+4)} \right) \, dx
            \end{equation*}
            \item We can then integrate the second integral and simplify.
            \begin{equation*}
                = \frac{1}{2} \ln(x+2) - \frac{1}{2} \ln(x+4) + C
            \end{equation*}
            \item We can then bring this back to the original integral.
            \begin{equation*}
                \int 1 \, dx -2 \int \frac{3x+4}{x^2+6x+8} \, dx = x -3 \ln(|x^2+6x+8|) - 5 \ln(|x+4|) + 5 \ln(|x+2|) + C
            \end{equation*}
            \begin{equation*}
                = x -3 \ln(|x+2| \, |x+4|) + 5 \ln(|x+2|) - 5 \ln(|x+4|) + C
            \end{equation*}
        \end{itemize}
    \end{enumerate}

    \newpage
    % Question 4

    \section*{Question 4}

    \item The development of the population $P$ in specific small city is analysed. The investigation
    revealed that the rate of the change of population per year can be modeled as:
    \begingroup
    % Make math size larger
    \large
    \begin{equation*}
        P'(t) = \frac{10000}{(t+2)^2}
    \end{equation*}
    \endgroup
    where $t$ is the time in years $t$ from today. What is the expected difference from today's population
    in the long run (this is for $t \to \infty$)?
    \begin{itemize}[label={}]
        \item We can start by integrating this function between the time now ($t=0$) and the time in the long run ($t \to \infty$).
        \begin{equation*}
            \int_{0}^{\infty} \frac{10000}{(t+2)^2} \, dt
        \end{equation*}
        \item We can then integrate the function.
        \begin{equation*}
            \begin{aligned}
                \int_{0}^{\infty} \frac{10000}{(t+2)^2} \, dt &= \int_{0}^{\infty} 10000(t+2)^{-2} \, dt \\
                &= \left[ -10000(t+2)^{-1} \right]_{0}^{\infty} \\
                &= \left[ -\frac{10000}{t+2} \right]_{0}^{\infty} \\
            \end{aligned}
        \end{equation*}
        \item We let $k=t$ and $k=0$ as $t \to \infty$.
        \begin{equation*}
            \begin{aligned}
                &= \left(\lim_{k \to \infty}  -\frac{10000}{k+2} \right) - \left( -\frac{10000}{0+2} \right) \\
                &= 0 - \left( -\frac{10000}{2} \right) \\
                &= \frac{10000}{2} \\
                &= 5000 \\
            \end{aligned}
        \end{equation*}
        \item Therefore, the expected difference from today's population in the long run is $5000$ people.
    \end{itemize}

    \newpage
    % Question 5

    \section*{Question 5}

    \item Find the area of the region bounded by the curved $y=x^2-1$ and $y=2x+2$.
    \begin{itemize}[label={}]
        \item We can find the area of the region bounded by the curves by integrating the difference between the two curves. First, we can find the intersection points of the two curves.
        \item To find the intersection points, we can make each function equal to each other.
        \begin{equation*}
            \begin{split}
                x^2-1 &= 2x+2 \\
                x^2-2x-3 &= 0 \\
                (x-3)(x+1) &= 0 \\
                x &= 3, -1 \\
            \end{split}
        \end{equation*}
        % Find intersection y values
        \begin{equation*}
            \begin{split}
                y &= (3)^2-1 \\
                &= 8 \\
                y &= (-1)^2-1 \\
                &= 0 \\
            \end{split}
        \end{equation*}
        So the intersection points are $(3,8)$ and $(-1,0)$.
        \item Now that we have found the intersection points, we can integrate the difference between the two curves.
        \item We can integrate the difference between the two curves from $x=-1$ to $x=3$.
        \begin{equation*}
            \begin{split}
                \int_{-1}^{3} (2x+2)-(x^2-1) \, dx &= \int_{-1}^{3} 2x+2-x^2+1 \, dx \\
                &= \left[ x^2 + 2x - \frac{x^3}{3} + x \right]_{-1}^{3} \\
                &= \left[ \frac{9}{3} + 6 - \frac{27}{3} + 3 - \left( 1 - 2 + \frac{1}{3} - 1 \right) \right] \\
                &= \frac{9}{3} + 6 - \frac{27}{3} + 3 - 1 + 2 - \frac{1}{3} + 1 \\
                &= 10.333 \\
            \end{split}
        \end{equation*}
        \item Therefore, the area of the region bounded by the curves is $10.333$ units squared.
    \end{itemize}

    \newpage
    % Question 6

    \section*{Question 6}

    \item Calculate the volume of the solid of revolution that is formed by revolving the curve $y=2+\sin (x)$ over the interval $[0, 2\pi]$ around the $x$-axis.
    \begin{itemize}[label={}]
        \item To calculate the volume of the trigonometric solid of revolution, we can use the formula:
        \begin{equation*}
            \displaystyle{V = \pi \int_{a}^{b} (f(x))^2 \, dx}
        \end{equation*}
        \item We can substitute the values into the formula.
        \begin{equation*}
            \begin{split}
                V &= \pi \int_{0}^{2\pi} (2+\sin (x))^2 \, dx \\
                &= \pi \int_{0}^{2\pi} 4 + 4\sin (x) + \sin^2 (x) \, dx \\
                &= \pi \int_{0}^{2\pi} 4 + 4\sin (x) + \frac{1}{2} - \frac{1}{2} \cos (2x) \, dx \\
                &= \pi \left[ 4x - 4\cos (x) + \frac{1}{2} x + \frac{1}{4} \sin (2x) \right]_{0}^{2\pi} \\
                &= \pi \left( \left( 8\pi - 4 + \frac{1}{2} 2\pi + \frac{1}{4}\right) - \left( -4 \right)   \right) \\
                &= \pi \left[ 9\pi -4 +4 \right] \\
                &= \pi \left[ 9\pi \right] \\
                &= 9\pi^2 \\
            \end{split}
        \end{equation*}
        \item Therefore, the volume of the solid of revolution is $9\pi^2$ units cubed.
    \end{itemize}

    \newpage
    % Question 7

    \section*{Question 7}
    
    \item Consider the function $f$ defined on $\mathbb{R}$ that fulfils the following conditions:
    \begin{equation*}
        f'(x) = D \cdot f(x) \text{ for all } x \in \mathbb{R} \text{ and } f(0) = f_0
    \end{equation*}
    where $D$ and $f_0$ are given non-zero constants. This type of equation arises in many applications when modelling decay $(D>0)$ or growth $(D<0)$. Your task is to find $f$.
    
    % Question 7a
    
    \begin{enumerate}
        \item Calculate the coefficients $c_n$ of the power series representation:
        \begin{equation*}
            f(x) = \sum_{n=0}^{\infty} c_n x^n
        \end{equation*}
        of the solution $f$.
        \begin{itemize}[label={}]
            \item \begin{equation*}
            f(a) = \sum_{n=0}^{\infty} c_n a^n
            \end{equation*}

            \item Now, take the derivative of both sides of the power series representation of $f(x)$:

            \begin{equation*}
            f'(x) = \sum_{n=1}^{\infty} n c_n x^{n-1}
            \end{equation*}

            \item Substituting $x=a$ into this expression, we get:

            \begin{equation*}
            f'(a) = \sum_{n=1}^{\infty} n c_n a^{n-1}
            \end{equation*}

            \item We can continue taking derivatives of both sides of the power series representation of $f(x)$ to obtain expressions for higher-order derivatives of $f(x)$ evaluated at $x=a$. Specifically, we have:

            \begin{align*}
            f''(x) &= \sum_{n=2}^{\infty} n(n-1) c_n x^{n-2} \\
            f''(a) &= \sum_{n=2}^{\infty} n(n-1) c_n a^{n-2} \\
            f'''(x) &= \sum_{n=3}^{\infty} n(n-1)(n-2) c_n x^{n-3} \\
            f'''(a) &= \sum_{n=3}^{\infty} n(n-1)(n-2) c_n a^{n-3} \\
            &\vdots
            \end{align*}

            \item We can use these expressions to solve for the coefficients $c_n$ using a system of equations. Specifically, we have:

            \begin{align*}
            f(a) &= c_0 \\
            f'(a) &= c_1 \\
            f''(a) &= 2c_2 \\
            f'''(a) &= 3 \cdot 2 \cdot c_3 \\
            &\vdots \\
            f^{(n)}(a) &= n! \cdot c_n \\
            &\vdots
            \end{align*}

            Solving this system of equations for the coefficients $c_n$, we get:

            \begin{equation*}
            c_n = \frac{f_0(D^n)}{n!}
            \end{equation*}

            where $f_0(D^n)$ denotes the $n$th derivative of $f(x)$.
        \end{itemize}

        % Question 7b

        \item Calculate the radius of convergence of the power series of part (a).
        \begin{itemize}[label={}]
            \item To calculate the radius of convergence, we can use the ratio test. 
            \item The ratio test in terms of $c_n x^n$ is:
            \begin{equation*}
                \lim_{n \to \infty} \left| \frac{c_{n+1} x^{n+1}}{c_n x^n} \right|
            \end{equation*}
            \item We can substitute $c_n$ into the ratio test.
            \begin{equation*}
                \begin{aligned}
                    \lim_{n \to \infty} \left| \frac{c_{n+1} x^{n+1}}{c_n x^n} \right| &= \lim_{n \to \infty} \left| \frac{\frac{f_0 (D^{n+1})}{(n+1)!} x^{n+1}}{\frac{f_0 (D^n)}{n!} x^n} \right| \\
                    &= \lim_{n \to \infty} \left| \frac{f_0 (D^{n+1}) x^{n+1}}{f_0 (D^n) x^n} \cdot \frac{n!}{(n+1)!} \right| \\
                    &= \lim_{n \to \infty} \left| \frac{Dx}{n+1} \right| \\
                    &= |Dx| \lim_{n \to \infty} \left| \frac{1}{n+1} \right| \\
                    &= |Dx| \cdot 0 \\
                    &= 0 \\
                \end{aligned}
            \end{equation*}
            \item Since the limit of the ratio test is $0$, it means that the series is convergent for all $x$.
            \item Therefore, the interval of convergence is $(-\infty, \infty)$ and the radius of convergence is $\infty$.
        \end{itemize}

        % Question 7c
        \newpage

        \item Use the Taylor series for the exponential function $e^t$ to derive the Taylor series of the function $g(x)=f_0e^{Dx}$ and compare the result with the power series you obtained in
        part (a).
        \begin{itemize}[label={}]
            \item We can start off by finding the Taylor series of the exponential function $e^t$.
            \begin{equation*}
                \begin{split}
                    e^t &= \sum_{n=0}^{\infty} \frac{t^n}{n!} \\
                    &= 1 + t + \frac{t^2}{2!} + \frac{t^3}{3!} + \frac{t^4}{4!} + \cdots \\
                \end{split}
            \end{equation*}
            We can then use this Taylor series to find the Taylor series of the function $g(x)$.
            \begin{equation*}
                \begin{split}
                    g(x) &= f_0 e^{Dx} \\
                    &= f_0 \sum_{n=0}^{\infty} \frac{(Dx)^n}{n!} \\
                    &= f_0 \left( 1 + Dx + \frac{(Dx)^2}{2!} + \frac{(Dx)^3}{3!} + \frac{(Dx)^4}{4!} + \cdots \right) \\
                    &= f_0 + f_0 Dx + \frac{f_0 (Dx)^2}{2!} + \frac{f_0 (Dx)^3}{3!} + \frac{f_0 (Dx)^4}{4!} + \cdots \\
                \end{split}
            \end{equation*}
            With this, we can see that the Taylor series of the function $g(x)$ is the same as the power series we obtained in part (a).
        \end{itemize}
    \end{enumerate}

    







\end{enumerate}
\end{document}