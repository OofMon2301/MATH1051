\documentclass[12pt]{article}
\usepackage[paper=letterpaper,margin=3.5cm]{geometry}
\usepackage{amsmath}
\usepackage{amssymb}
\usepackage{amsfonts}
\usepackage{newtxtext, newtxmath}
\usepackage{enumitem}
\usepackage{titling}
\usepackage{nccmath}
\usepackage[colorlinks=true]{hyperref}


\setlength{\droptitle}{-6em}

% Enter the specific assignment number and topic of that assignment below, and replace "Your Name" with your actual name.
\title{Assignment \# 4: MATH1051}
\author{Jamie Chen\\ \text{Student Number:} \texttt{48093189} \\ \text{Semester 2, 2023}}
\date{\today}

\begin{document}
\maketitle
\begin{enumerate}[leftmargin=\labelsep]
%% Question 1

    \item {\bf (1 mark each)} This is the main question. Determine the domains (as a subset of $\mathbb{R}$) of the functions
        \begin{enumerate}
            \item $f_1(x)=\frac{1}{e^x-e^{-x}}$
                \begin{itemize}[label={}]
                    \item 
                \end{itemize}
            \item $f_2(x)=\frac{1}{\sqrt{4-x^2}}$
                \begin{itemize}[label={}]
                    \item 
                \end{itemize}
            \item $f_3(x)= \log \arccos x$ 
                \begin{itemize}[label={}]
                    \item 
                \end{itemize}
        \end{enumerate}
            
%% New Page
\newpage
%% Question 2

    \item {\bf (3 marks)} Given is the function $g(x)=x^2+3x$. For a second function $f$ with $f(3)=0$ we find $(g \circ f)(x)=x^2-3x$. What is the function $f$? Is $f$ unique?
        \begin{itemize}[label={}]
            \item This is body text
        \end{itemize}

%% New Page
\newpage        
%% Question 3
    
    \item {\bf (1 mark each)} Determine which of the following functions are 1-1? Prove your answer.
        \begin{enumerate}
            \item
                \begin{itemize}[label={}]
                    \item 
                \end{itemize}
            \item 
                \begin{itemize}[label={}]
                    \item 
                \end{itemize}
            \item 
                \begin{itemize}[label={}]
                    \item 
                \end{itemize}
        \end{enumerate}
            
%% New Page
\newpage
%% Question 4 

    \item {\bf (1 mark each)} Determine what the following limits are or show that they do not exist.
        \begin{enumerate}
            \item 
                \begin{itemize}[label={}]
                    \item 
                \end{itemize}
            \item 
                \begin{itemize}[label={}]
                    \item 
                \end{itemize}
            \item 
                \begin{itemize}[label={}]
                    \item 
                \end{itemize}
            \item 
                \begin{itemize}[label={}]
                    \item 
                \end{itemize}
            \item 
                \begin{itemize}[label={}]
                    \item 
                \end{itemize}
        \end{enumerate}

%% New Page
\newpage
%% Question 5

    \item {\bf (1 mark each)} Consider the sequence $a_n$ defined by the recursion
        \begin{equation}
            a_n=a_{n-1}-\frac{1}{4}a_{n-2}
        \end{equation} for $n=3,4,5,\dots$.
        \begin{enumerate}
            \item Calculate $h$ such that $a_n=h^{n-1}$ fulfils the recursive definition.
                \begin{itemize}[label={}]
                    \item 
                \end{itemize}
            \item What is the limit of the sequence $a_n$ (if it exists)?
                \begin{itemize}[label={}]
                    \item 
                \end{itemize}
        \end{enumerate}

%% New Page
\newpage
%% Question 6

    \item {\bf (1 mark each)} Given is the sequence $b_n$ defined in recursive form
        \begin{equation*}
            b_n=\frac{1}{2}\left(b_{n-1}+\frac{A}{b_{n-1}} \right)
        \end{equation*} for a given $A>0$. You can assume that all values of $b_n$ are non-zero.
        \begin{enumerate}
            \item Use your calculator (or MATLAB) to calculate the first four values of the sequence $b_n$ starting from $b_1=A$ (this is for $n=1,2,3,4$). Inspecting these values do you expect the sequence to be convergent or to be divergent?
                \begin{itemize}[label={}]
                    \item 
                \end{itemize}
            \item Assume you know the sequence $b_n$ is converging, what would be its limit (or its limits)? Justify your answer. Is it consistent with your result of part (a)?
                \begin{itemize}[label={}]
                    \item 
                \end{itemize}
        \end{enumerate}

%% New Page
\newpage
%% Question 7

    \item {\bf (1 mark each)} Assume you have given a sequence $c_n$ with non-zero values ($c_n\neq0$ for $n=1,2,\dots$) that fulfils the condition
        \begin{equation}
            \left|\frac{c_n}{c_n-1}\right|\leq q
        \end{equation} for all $n=1,2,3,\dots$ for some fixed constant q with $0<q<1$.
        \begin{enumerate}
            \item Show that $c_n\to0$ for $n\to \infty$. (Hint: use the squeeze theorem)
                \begin{itemize}[label={}]
                    \item 
                \end{itemize}
            \item Use the result from part(a) to show that $\displaystyle{\lim_{n \to \infty}}\,\,\, \frac{2^n}{n!}=0$.
                \begin{itemize}[label={}]
                    \item 
                \end{itemize}
            \item Again: use the result from part(a) to show that $\displaystyle{\lim_{n \to \infty}}\,\,\, \frac{1}{3^nn^3}=0$.
        \end{enumerate}


\end{enumerate}
\end{document}