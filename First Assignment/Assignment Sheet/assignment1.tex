\documentclass[12pt,english]{article}
\usepackage[T1]{fontenc}
\usepackage[latin9]{inputenc}
\usepackage{geometry}
\geometry{verbose,tmargin=2cm,bmargin=2cm,lmargin=2cm,rmargin=2cm}
\usepackage{fancyhdr}
\pagestyle{fancy}
\usepackage{babel}
\usepackage{verbatim}
\usepackage{float}
\usepackage{amssymb}
\usepackage{amsmath}
\usepackage{graphicx}
\usepackage{setspace}
\onehalfspacing
\usepackage[unicode=true]{hyperref}

\newcommand{\ds}{\displaystyle}

\makeatletter

%%%%%%%%%%%%%%%%%%%%%%%%%%%%%% LyX specific LaTeX commands.
%% Because html converters don't know tabularnewline
\providecommand{\tabularnewline}{\\}

%%%%%%%%%%%%%%%%%%%%%%%%%%%%%% User specified LaTeX commands.
\lhead{MATH1051 - Semester 2 2023}
\chead{Assignment I}
\rhead{Due: {\bf Friday 18 August, 3 pm.}}

\makeatother

\begin{document}


\begin{itemize}
\item Write your answers clearly. Illegible assignments will not be marked. 
\item Show all your workings. Correct answers without justification will
not receive full marks.
\item Wherever possible, answers should be given in exact form.
\item This assignment is worth 10\% of the total assessment for the course.

\item Submit your assignment as a {\bf single PDF} via the Assignment submission
link on Blackboard.
\item Submit all applications for extensions via the my.UQ portal. 
\item Marking: The total mark is $25$.
\begin{itemize}
\item Marking Scheme for questions worth 1 mark:
\begin{itemize}
\item Mark of 0: You have not submitted a relevant answer, or you have no
strategy present in your submission. 
\item Mark of 1/2: You have the right approach, but need to fine-tune some
aspects of your justification/calculations. 
\item Mark of 1: You have demonstrated a good understanding of the topic
and techniques involved, with clear justification and well-executed
calculations. 
\end{itemize}
\item Marking Scheme for questions worth 3 marks:
\begin{itemize}
\item Mark of 0: You have not submitted a relevant answer, or you have no
strategy present in your submission. 
\item Mark of 1: Your submission has some relevance, but does not demonstrate
deep understanding or sound mathematical technique. This topic needs
more attention! 
\item Mark of 2: You have the right approach, but need to fine-tune some
aspects of your justification/calculations. 
\item Mark of 3: You have demonstrated a good understanding of the topic
and techniques involved, with clear justification and well-executed
calculations. 
\end{itemize}
\pagebreak{}
\end{itemize}
\end{itemize}
{\large{}Attach this page to the front of your submission. Remember
to sign the declaration.\bigskip{}
}{\large\par}
\begin{quote}
\begin{flushleft}
I hereby state that the work contained in this assignment has not
previously been submitted for assessment, either in whole or in part,
by either myself or any other student at either The University of
Queensland or at any other tertiary institution except where explicitly
acknowledged. To the best of my knowledge and belief, the assignment
contains no material that has been previously published or written
by another person except where due reference is made. I make this
Statement in full knowledge of an understanding that should it be
found to be false, I will be subject to disciplinary action under
Student Integrity and Misconduct Policy 3.60.04 of the University
of Queensland. The University of Queensland's policy on plagiarism
can be found at \href{http://ppl.app.uq.edu.au/content/3.60.04-student-integrity-and-misconduct}{http://ppl.app.uq.edu.au/content/3.60.04-student-integrity-and-misconduct}
(Reference 3.60.04).\\
\vspace{2cm}
Name .......................................................................Student
ID .........................\vspace{2cm}
\par\end{flushleft}

\noindent Signed .......................................................................Date
.................................\vspace{0.5cm}

\end{quote}


\newpage{}

\vspace{4cm}

\begin{itemize}
\item[(1)]
Determine the domains (as subset of $\mathbb{R}$)  of the functions \hfill{(1 mark each)}
\begin{itemize}
    \item[(a)] $f_1(x)=\frac{1}{e^{x}-e^{-x}}$,
    \item[(b)] $f_2(x)=\frac{1}{\sqrt{4-x^2}}$.
    \item[(c)] $f_3(x)=\log{\arccos{x}}$.
    
\end{itemize}
\item[(2)]
Given is the function $g(x)=x^2+3x$. For a second function $f$ with $f(3)=0$ we find $(g \circ f)(x) = x^2-3x$. 
What is the function $f$? Is $f$ unique?
\hfill{(3 mark)}


\item[(3)]
Determine which of the following functions is 1-1? Prove your answer. \hfill{(1 mark each)}
\begin{itemize}
\item[(a)] $f_1(x)=e^{-x^2}$
\item[(b)] $f_2(x)=2x^2-3x+1$
\item[(c)] $f_3(x)=|x|+2 \cdot x$
\end{itemize}

\item[(4)]
Determine what the following limits are or show that they do not exist: \hfill{(1 mark each)}
\begin{itemize}
\item[(a)] $ {\lim \limits_{{n\rightarrow \infty}}} \frac{(n^2+4 n-27)(n^3-1)}{(n(n-1))^2} $
\item[(b)] $\lim \limits_{n\rightarrow \infty} \frac{3n^2-9n + 48}{4n^3}$
\item[(c)] $\lim \limits_{n\rightarrow \infty} \frac{(3n+1)^3-27n^3}{n^2}$
\item[(d)] $\lim \limits_{n\rightarrow \infty} \frac{2n^2}{2n-1}-n$
\item[(e)] $\lim \limits_{n\rightarrow \infty} \sqrt{n(n+1)}-n$
\end{itemize}

\vfill\hspace*{\fill}{\em Turn over for more questions}

\newpage

\item[(5)] Consider the sequence $a_n$ defined by the recursion 
\begin{equation}
a_n=a_{n-1}-\frac{1}{4}a_{n-2}
\label{EQ.IV}    
\end{equation}
for $n=3,4,5,\ldots$. \hfill{(1 mark each)}
\begin{itemize}
    \item[(a)] Calculate $h$ such that $a_n=h^{n-1}$ fulfills the recursive definition~\ref{EQ.IV}.  
\item[(b)]What is the limit of the sequence $a_n$ (if it exists)?
\end{itemize}


\item[(6)] Given is a sequence $b_n$ defined in recursive form 
$$
b_n=\frac{1}{2}\left(b_{n-1}+\frac{A}{b_{n-1}}\right)
$$
for a given $A>0$. You can assume that all values of $b_n$ are non-zero.
\begin{itemize}
    \item[(a)] For $A=2$ use your calculator (or MATLAB) to calculate the first four values of the sequence $b_n$ starting from $b_1=A$ (this is for $n=1,2,3,4$). Inspecting these values: do you expect the sequence to be convergent or to be divergent?  \hfill{(1 mark)}
    \item[(b)] Assume you know the sequence $b_n$ is converging, what would be its limit (or its limits)? Justify your answer. Is it consistent with your result of part (a)? \hfill{(3 marks)}
\end{itemize}


\item[(7)] Assume you have given a sequence $c_n$
with non-zero values ($c_n \not =0$ for $n=1,2,\ldots$)
that fulfils the condition 
\begin{equation}
\left|\frac{c_n}{c_{n-1}}\right| \le q    
\end{equation}
for all $n=1,2,3,\ldots$ for some fixed constant $q$ with $0< q <1$.
\begin{itemize}
    \item[(a)] Show that $c_n \to 0$ for $n \to \infty$. (Hint: use the squeeze theorem) \hfill{(3 marks)}
    \item[(b)] Use the result from part(a) to show that
$\lim \limits_{n\rightarrow \infty} \frac{2^n}{n!} = 0$ \hfill{(1 mark)}
    \item[(b)] Again: use the result from part(a) to show that
$\lim \limits_{n\rightarrow \infty} \frac{1}{3^n n^3} = 0$ \hfill{(1 mark)}
\end{itemize}
\end{itemize}
\vspace{4.5cm}

\center{{\bf END OF ASSIGNMENT}}

\end{document}
